\documentclass[conference,10pt]{IEEEtran}

%% IEEE CNS addition:
\makeatletter
\def\ps@headings{%
	\def\@oddhead{\mbox{}\scriptsize\rightmark \hfil \thepage}%
	\def\@evenhead{\scriptsize\thepage \hfil \leftmark\mbox{}}%
	\def\@oddfoot{}%
	\def\@evenfoot{}}
\makeatother
\pagestyle{empty}

\usepackage[dvipsnames]{xcolor}
\usepackage{graphicx}

\usepackage[binary-units]{siunitx}
\sisetup{range-phrase=--, range-units=single}

\usepackage[basic]{complexity}
\usepackage[super,negative]{nth}

\usepackage{cleveref}
\newcommand{\crefrangeconjunction}{--}

\usepackage{booktabs}
\usepackage{microtype}

%% Fix indent in new section...
\newcommand{\subparagraph}{}
\usepackage{titlesec}
\titlespacing*{\section}{0pt}{1.5ex}{0.7ex}

%bib
\usepackage[maxnames=3,maxbibnames=99,mincrossrefs=5,style=ieee,sortcites,backend=bibtex]{biblatex}
%\addbibresource{papers-off.bib}
%\addbibresource{confs-off.bib}
%\addbibresource{rfc.bib}

%picky abt et al.
\usepackage{xpatch}

\xpatchbibmacro{name:andothers}{%
	\bibstring{andothers}%
}{%
	\bibstring[\emph]{andothers}%
}{}{}

%opening!

\newcommand{\mytitle}{%Dynamically Annealed Reinforcement Learning for Intrusion Prevention
Dynamic Exploration via the Wavelet Transform}

\usepackage{hyperref}
\usepackage{url}
\hypersetup{
	colorlinks,
	citecolor=black,
	filecolor=black,
	linkcolor=black,
	urlcolor=black,
	pdftitle={\mytitle{}},
	pdfauthor={Kyle A. Simpson}
}
\newcommand*{\email}[1]{\href{mailto:#1}{\nolinkurl{#1}} } 

\usepackage{titling}
\settowidth{\thanksmarkwidth}{*}
\setlength{\thanksmargin}{-\thanksmarkwidth}

%% Enable /thanks
\IEEEoverridecommandlockouts
\makeatletter
\def\footnoterule{\relax%
	\kern-5pt
	\hbox to \columnwidth{\hfill\vrule width 0.5\columnwidth height 0.4pt\hfill}
	\kern4.6pt}
\makeatother

%-------------------------------------%
%-------------------------------------%

\title{\mytitle{}}
\author{Kyle A. Simpson\thanks{This work was supported by the Engineering and Physical Sciences
		Research Council [grant number EP/M508056/1]},\\\emph{University of Glasgow, Glasgow, Scotland},\\
		\email{k.simpson.1@research.gla.ac.uk}}

% Remove date, leave no spacing.
\predate{}
\postdate{}
\date{}

\begin{document}

%% If needed, make urls typewritery
%\urlstyle{tt}

\maketitle

\begin{abstract}
	Make it look convincing!
\end{abstract}

\section{Introduction}

Okay here is a... paper?

\section{Initial Definitions}

Testing to see if I get the right header behaviour.
Another sentence to bring out the edge cases in this style.

Does next paragraph have proper indent? Please!

\section{A Plan, of Sorts}

\begin{enumerate}
	\item Establish criteria used to inform system temp---what could make my system increase $\epsilon$ again late into training?
	\begin{enumerate}
		\item Current thoughts: Wavelet analysis of the reward signal.
		\item Intuition: multiscale representation of variation in reward.
		\item What do we see? Hopefully, state-to-state variation as well as longer-term characteristics.
		\item Expect that overall worsening on the coarsest scales should indicate pressing, global need for exploration  (i.e., non-stationarity, changing problem).
		\item We can perhaps derive a different uncertainty for each scale -- how do we choose? Combine answers, or (in turn) randomly select an $\epsilon$ from the derived set?
	\end{enumerate}
	\item What are my chosen features? Reward Function?
	\item What actions do I give the agents? Just more probabilistic packet dropping?
	\item Test against MARL and others---probably essential.
	\item Model at least three scenarios: standard (under attack), normality shifts, attack changes.
	\item \textsc{Ideally:} test adversarial work (what do such examples look like for my classifier, can an attacker actually operate on them?). How do results change with/without pretraining?
\end{enumerate}

\printbibliography

\end{document}