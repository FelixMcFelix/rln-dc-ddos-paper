\documentclass[10pt, a4paper]{article}

\usepackage[dvipsnames]{xcolor}
\usepackage{graphicx}

\usepackage[binary-units]{siunitx}
\sisetup{range-phrase=--, range-units=single}

\usepackage[basic]{complexity}
\usepackage[super,negative]{nth}

\usepackage{booktabs}
\usepackage{microtype}

%% Fix indent in new section...
\usepackage{titlesec}
\titlespacing*{\section}{0pt}{1.5ex}{0.7ex}

%bib
\usepackage[maxnames=3,maxbibnames=99,mincrossrefs=5,style=ieee,sortcites,backend=bibtex]{biblatex}
\addbibresource{report.bib}

%picky abt et al.
\usepackage{xpatch}

\xpatchbibmacro{name:andothers}{%
	\bibstring{andothers}%
}{%
	\bibstring[\emph]{andothers}%
}{}{}

%opening!

\newcommand{\mytitle}{Lab Report on reimplementation of MARL}

\usepackage{varioref}
\usepackage{hyperref}
\usepackage{url}
\hypersetup{
	colorlinks,
	citecolor=black,
	filecolor=black,
	linkcolor=black,
	urlcolor=black,
	pdftitle={\mytitle{}},
	pdfauthor={Kyle A. Simpson}
}
\usepackage{cleveref}
\newcommand{\crefrangeconjunction}{--}

\newcommand*{\email}[1]{\href{mailto:#1}{\nolinkurl{#1}} } 

\usepackage{titling}
\settowidth{\thanksmarkwidth}{*}
\setlength{\thanksmargin}{-\thanksmarkwidth}

%-------------------------------------%
%-------------------------------------%

\title{\mytitle{}}
\author{Kyle A. Simpson}

\begin{document}

%% If needed, make urls typewritery
%\urlstyle{tt}

\maketitle

\section{Introduction}

Goal: reimplementation of \textcite{DBLP:journals/eaai/MalialisK15}.

\section{Empirical study}

Rough hypotheses:
\begin{itemize}
	\item The findings of \citeauthor{DBLP:journals/eaai/MalialisK15} are replicable in a `real' network environment.
	
	\item Distributed MARL is tested against an abnormal (ISP-like) network where the network administrator is assumed to have control of nodes which are local to few hosts.
	DDoS mitigation via a `direct-control' reinforcement learning approach over blanket pushback will be expected to cause significant collateral damage, and accordingly have significantly lower performance.
	
	\item Learners taught with a gradually/linearly annealed $\epsilon$ in this problem domain will be unable to generalise to the case where host distribution, connection patterns etc.\ exhibit explicit non-stationarity.
	\begin{itemize}
		\item For instance, changepoints at later episodes concerning attacker distribution, host distribution, load profiles...
	\end{itemize}
\end{itemize}

\subsection{Methodology}

Yeahh..

\paragraph{Network topology.}
As in the paper. (describe here? TikZ graph?)

?? Simple enough flow rules, just send up the tree indiscriminately. The model doesn't concern the server having to reply to its clients, mysteriously. Oh well.

\paragraph{Traffic generation.}
Traffic is generated using \emph{TCPReplay} at each host, playing back traffic indefinitely from a predefined Pcap file.
??Control over the send rate is managed by either TC, or tcpreplay. I'm having trouble with both at the moment!

The MARL network model requires that hosts be designated `good' or `bad'.
Calculating the reward function for the current network state requires knowledge of which traffic is (il)legitimate, by heuristic or otherwise.
As \citeauthor{DBLP:journals/eaai/MalialisK15} assume perfect knowledge in their training process, I rewrite the traffic from each host to encode this in the last octet of the source IP address.
The scheme is simple: if the octet is even, the packet is `good' (and vice versa).

\paragraph{Pushback.}
Mininet's switches run \emph{Open vSwitch} (OvS), allowing them to be remotely configured via \emph{OpenFlow} messages.
To enable pushback via probabilistic packet dropping as in the original experiment, OvS was modified to include a custom action to do just this for any given flow.

Typically, OvS switches register with a single controller: when a packet which can't be routed arrives, the switch forwards this to the controller, which is responsible for replying with new rules to make delivery possible.
The control application acts outside of the typical Switch--Controller relationship: given that rule updates must be periodically posted \emph{at the control application's discretion}, the flow updates are manually composed and sent directly.

\paragraph{Load monitoring.}
All switches in Mininet appear to occupy the same namespace, and so a daemon running on any Mininet switch is allowed to monitor the interfaces of any other.
This fact is used to easily monitor each interface in a centralised manner.

The monitoring daemon spawns one thread for each interface it's told to subscribe to.
Each of these threads then handles the \emph{libpcap} events attached to that interface, determining whether a packet is `good' or `bad' and appending the packet length to a dedicated location in a shared store.
Every timestep, the calling program locks the shared store and reads out the time difference, traffic counts, and classifications for each interface.
These statistics enable a rough calculation of the current load in \si{\mega\bit\per\second}.

\paragraph{Reinforcement learning.} Opportunistic SARSA! ($\gamma = 0$, no discounting)

For an action $a-t$ chosen by the current value function $Q_t$ while in state $s_t$ ??CLEANUP:

$$ Q_{t+1}(s_t,a_t) = Q_t(s_t,a_t) + \alpha [R_{t+1} + \gamma Q_t(s_{t+1},a_{t+1}) - Q_t(s_t,a_t)] $$
Actions received from each learner cause a different level of throttling to be applied within the network.

?? Tabular approximation, tile coding, maybe list the params here?!

\paragraph{Changes for experiments 2... 3...} Undecided.

\subsection{Results}

TBA. Currently awaiting debugging of anomalous mininet behaviour.

\section{Discussion}

Yeah, those sure were some results huh.

\section{Conclusion}

Yikes!

\printbibliography

\end{document}